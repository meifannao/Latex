% !TEX TS-program = xelatex
% !TEX encoding = UTF-8 Unicode
% !Mode:: "TeX:UTF-8"

\documentclass{resume}
\usepackage{zh_CN-Adobefonts_external} % Simplified Chinese Support using external fonts (./fonts/zh_CN-Adobe/)
% \usepackage{NotoSansSC_external}
% \usepackage{NotoSerifCJKsc_external}
% \usepackage{zh_CN-Adobefonts_internal} % Simplified Chinese Support using system fonts
\usepackage{linespacing_fix} % disable extra space before next section
\usepackage{cite}

\usepackage{framed}
\usepackage{color}  % 改变背景颜色
\definecolor{shadecolor}{rgb}{0.92,0.92,0.92} % 灰色

\makeatletter
\@namedef{ver@everypage.sty}{9999/99/99}
\makeatother
\usepackage{everypage-1x}
\usepackage[contents=DRAFT, color=red, opacity=0.2]{background}
\backgroundsetup{scale=0.4, angle=0, opacity = 0.5,  % 放缩  旋转  透明度
contents = {\includegraphics[width=\paperwidth, height=\paperwidth, keepaspectratio] {time}}}  % 添加背景图片

\usepackage{multirow}

\begin{document}
\pagenumbering{gobble} % suppress displaying page number

\begin{tabular}{c c|l}

\multirow[c]{3}{3.7in}[-0.05in]{\Huge\fangzheng 彭于晏}
&  \multirow[c]{3}{0.7in}[0.28in]{ \includegraphics[width=0.7in]{time}}
&  \phone{(+86) 18574791536} \\
& &  \qq{(QQ) 1973355104} \\
& &  \email{(mail) meifannao549@gmail.com}\\

\end{tabular}
 
\begin{shaded} 
\section{\faGraduationCap \fangzheng \  教育背景}
\datedsubsection{\textbf{xxxxx大学}}{2019 -- 至今}

\textit{-学院:计算机科学与工程学院}\\
\textit{-专业:计算机科学与技术(卓)}\\
\textit{-绩点:\textup{4.18}}\\
\textit{-排名:\textup{3} / \textup{37}}\\
\textit{-主修课程:计算机网络(卓):\textup{97},操作系统:\textup{98},数据结构:\textup{97}}\\
\textit{\hspace*{1.2cm} \qquad \textup{C}语言程序设计:\textup{98},面向对象程序设计:\textup{98}}


%\datedsubsection{\textbf{xxx大学} (xx-xx)}{2014 -- 2018}
%\textit{本科}\ --- xxx
\end{shaded}

\section{\faUsers \fangzheng \ 科研/兴趣}
%\datedsubsection{\textbf{竞赛经历}}{2019年 -- 至今}
%\role{获奖}
%
%国奖
%\begin{itemize}
%  \item 团体程序设计天梯赛获得个人国家三等奖
%  \item 团体程序设计天梯赛获得团体国家三等奖
%\end{itemize}
%
%省奖
%\begin{itemize}
%	\item 蓝桥杯程序设计大赛获得个人省赛一等奖
%	\item 高教社全国数学建模大赛省赛二等奖
%	\item 团体程序设计天梯赛获得团体省赛二等奖
%\end{itemize}
%校赛
%\begin{itemize}
%	\item ACM-ICPC程序设计校赛获得校赛一等奖
%	\item 数学建模校赛选拔赛二等奖
%\end{itemize}

\datedsubsection{\textbf{小项目}}{2020年9月 -- 至今}
\role{C++, Java, python}{个人项目}
\begin{onehalfspacing}
项目详细内容在请查看Github主页 \href{https://github.com/meifannao}{https://github.com/meifannao}
\begin{itemize}
  \item 基于拓扑排序的教学计划编制问题
  \item 基于Android Studio开发的安卓推箱子小游戏
  \item 基于Packer Tracer的网络拓扑设计
  \item 使用\LaTeX 制作的个人简历模板
  \item OpenVINO部署加速视频人像分割模型
  
\end{itemize}
\end{onehalfspacing}

%\datedsubsection{\textbf{\LaTeX\ 简历模板}}{2015 年5月 -- 至今}
%\role{\LaTeX, Python}{个人项目}
%\begin{onehalfspacing}
%优雅的 \LaTeX\ 简历模板, https://github.com/billryan/resume
%\begin{itemize}
%  \item 容易定制和扩展
%  \item 完善的 Unicode 字体支持,使用 \XeLaTeX\ 编译
%  \item 支持 FontAwesome 4.5.0
%\end{itemize}
%\end{onehalfspacing}

% Reference Test
%\datedsubsection{\textbf{Paper Title\cite{zaharia2012resilient}}}{May. 2015}
%An xxx optimized for xxx\cite{verma2015large}
%\begin{itemize}
%  \item main contribution
%\end{itemize}


\section{\faHeart \fangzheng \ 获奖情况}
\textbf{国赛:}
\datedline{\textit{三等奖}, 团体程序设计天梯赛个人赛}{2021 年\hspace*{0.195cm} 4 月}
\datedline{\textit{三等奖}, 团体程序设计天梯赛团体赛}{2021 年\hspace*{0.195cm} 4 月}
\textbf{省赛:}
\datedline{\textit{一等奖}, 蓝桥杯程序设计大赛}{2021 年\hspace*{0.195cm} 4 月}
\datedline{\textit{二等奖}, 高教社全国数学建模大赛}{2021 年\hspace*{0.195cm}  9 月}
\datedline{\textit{二等奖}, 计算机设计大赛--物联网赛道}{ 2022 年\hspace*{0.195cm}   5 月}
\datedline{\textit{三等奖}, 陕西省第九届ACM-ICPC大赛}{ 2022 年\hspace*{0.195cm}   5 月}
\textbf{其他:}
\datedline{\textit{奖学金},蔚蓝奖学金}{2020 年12 月}
\datedline{\textit{荣誉称号}, 尚真笃学学习先进个人}{2021 年12 月}
\datedline{\textit{五四评优}, 优秀共青团团员}{2021 年\hspace*{0.195cm} 4 月}
\datedline{\textit{社会实践}, "三下乡"社会实践优秀论文}{2021 年\hspace*{0.195cm} 12 月}
\datedline{\textit{科技竞赛}, 创新成果奖一等奖}{2022 年\hspace*{0.195cm} 5 月}
\datedline{\textit{社团活动}, 迎新晚会"优秀演员"}{2019 年\hspace*{0.195cm} 9 月}

\section{\faCogs \fangzheng \ IT 技能}
% increase linespacing [parsep=0.5ex]
\begin{itemize}[parsep=0.5ex]
  \item 编程语言: C++, Python, C, Java
  \item 辅助软件: \LaTeX,cmake,Makedown
  \item 工具: git,visio,Tikz
\end{itemize}

\section{\faInfo \fangzheng \ 其他}
% increase linespacing [parsep=0.5ex]
\begin{itemize}[parsep=0.5ex]
  \item  GitHub: \href{https://github.com/meifannao}{https://github.com/meifannao}
\end{itemize}

%\section{\faInfo \fangzheng \ 自我评价}
%
%\begin{itemize}
%	\item 较强的自学能力,扎实的数学基础,较强的逻辑思维能力
%	\item 较强的动手实践能力,课程设计均为优秀,担任ACM校队队长,编程能力较好
%	\item 善于与人交流,参加迎新晚会获得优秀演员称号,获得优秀共青团员称号
%\end{itemize}


%% Reference
%\newpage
%\bibliographystyle{IEEETran}
%\bibliography{mycite}
\end{document}
